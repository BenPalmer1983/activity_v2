\documentclass[12pt,twoside]{manual}

\pagestyle{plain}
\begin{document}

%%######################################################################
%% Styles
%%######################################################################

\tikzstyle{parentIsotope} = [rectangle, rounded corners, minimum width=3cm, minimum height=1cm,text centered, draw=black, fill=white]
\tikzstyle{unstableIsotope} = [rectangle, minimum width=3cm, minimum height=1cm,text centered, draw=black, fill=white]
\tikzstyle{stableIsotope} = [rectangle, rounded corners, minimum width=3cm, minimum height=1cm,text centered, draw=black, fill=white]
\tikzstyle{isotopeProduction} = [rectangle, minimum width=3cm, minimum height=1cm,text centered, draw=black, fill=white]




%%######################################################################
%% Cover Page
%%######################################################################

\begin{titlepage}
  \begin{center}
    \centerline{\includegraphics[width=0.7\textwidth]{images/Cover_Art}}


    \textbf{Department of Metallurgy \& Materials}

    \vspace*{2.0cm}
    \Large{}
    \textbf{Activity V2 Manual}
    \vspace{0.8cm}
    \normalsize{}

  \end{center}
\end{titlepage}

\pagenumbering{gobble}

\pagenumbering{roman} 

\tableofcontents

\pagenumbering{arabic}




%%######################################################################
%% Overview
%%######################################################################

\chapter{Overview}

\section{Reason for the Program}

The Activity program calculates how radioactive a target becomes after being irradiated by high energy ions.  It uses the TENDL-2019 database that contains cross-section data for protons and deuterons, and the JEFF-3.3 Radioactive Decay Data File.

\section{Program Requirements}

The user must provide a exyz file from the SRIM ion transport program and a breakdown of the composition of the target, as well as other irradiation parameters.



%%######################################################################
%% Installation
%%######################################################################

\chapter{Installation}

The program needs Python 3 installed in order to run.  At the time of writing, it has only been developed to run on a Linux operating system, but it shouldn't require much adjusting to run on a Windows computer too.

Download and install the latest version of Python 3:

https://www.python.org/downloads















\end{document}
